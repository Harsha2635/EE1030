%iffalse
\let\negmedspace\undefined
\let\negthickspace\undefined
\documentclass[journal,,12pt,onecolumn]{IEEEtran}
\usepackage{cite}
\usepackage{amsmath,amssymb,amsfonts,amsthm}
\usepackage{algorithmic}
\usepackage{graphicx}
\usepackage{textcomp}
\usepackage{xcolor}
\usepackage{txfonts}
\usepackage{listings}
\usepackage{enumitem}
\usepackage{mathtools}
\usepackage{gensymb}
\usepackage{comment}
\usepackage[breaklinks=true]{hyperref}
\usepackage{tkz-euclide} 
\usepackage{listings}
\usepackage{gvv}                                        
%\def\inputGnumericTable{}                                 
\usepackage[latin1]{inputenc}                                
\usepackage{color}                                            
\usepackage{array}                                            
\usepackage{longtable}                                       
\usepackage{calc}                                             
\usepackage{multirow}                                         
\usepackage{hhline}                                           
\usepackage{ifthen}                                           
\usepackage{lscape}
\usepackage{tabularx}
\usepackage{array}
\usepackage{float}
\usepackage{multicol}



\newtheorem{theorem}{Theorem}[section]
\newtheorem{problem}{Problem}
\newtheorem{proposition}{Proposition}[section]
\newtheorem{lemma}{Lemma}[section]
\newtheorem{corollary}[theorem]{Corollary}
\newtheorem{example}{Example}[section]
\newtheorem{definition}[problem]{Definition}
\newcommand{\BEQA}{\begin{eqnarray}}
\newcommand{\EEQA}{\end{eqnarray}}
\newcommand{\define}{\stackrel{\triangle}{=}}
\theoremstyle{remark}
\newtheorem{rem}{Remark}

% Marks the beginning of the document
\begin{document}
\bibliographystyle{IEEEtran}
\vspace{3cm}

\title{25th February, 2021\\Shift-2}
\author{EE24BTECH11063 - Y.Harsha Vardhan Reddy}
\maketitle

\bigskip

\renewcommand{\thefigure}{\theenumi}
\renewcommand{\thetable}{\theenumi}


\section*{Integer Type}
\begin{enumerate}
 \item $\lim \limits_{x\to 0} \frac{ax-\brak{e^{4x}-1}}{ax\brak{e^{4x}-1}} $\\
 exists and is equal to b, then the value of $a-2b$ is
 \item A line is a common tangent to the circle $\brak{x-3}^2+y^2=9$ and the parabola $y^2=4x$. If the two points of contact \brak{a,b} and \brak{c,d} are distinct and lie in the first quadrant, then 2\brak{a+c} is equal to
 \item The value of \\
 $\int_{-2}^{2} |3x^2-3x-6| \, dx$\\
 is
 \item If the remainder when x is divided by 4 is 3, then the remainder when $\brak{2020+x}^{2022}$ is divided by 8 is

 \item A line L passing through origin is perpendicular to the lines 
 \begin{align*}
     L_1:\vec{r}=\brak{3+t}\hat{i}+\brak{-1+2t}\hat{j}+\brak{4+2t}\hat{k}\\
     L_2:\vec{r}=\brak{3+2s}\hat{i}+\brak{3+2s}\hat{j}+\brak{2+s}\hat{k}
 \end{align*}
 If the co-ordinates of the point in the first octant on $L_2$ at the distance of $\sqrt{17}$ from the point of intersection of L and $L_1$ are \brak{a,b,c}, then 18\brak{a+b+c} is equal to
 \item A function $f$ is defined on \sbrak{-3,3} as\\
 \[
f(x) = 
\begin{cases} 
\min\{|x|, 2 - x^2\}, & -2 \leq x \leq 2 \\
\sbrak{|x|}, & 2 < |x| \leq 3 
\end{cases}
\]
where $\sbrak{x}$ denotes the greatest integer $\le$ x. The number of points, where $f$ is not differentiable in $\brak{-3,3}$ is
 
 
 \item If the curves $x=y^4$ and $xy=k$ cut at right angles, then $\brak{4k}^6$ is equal to
 \item The total number of two digit numbers 'n', such that $3^n+7^n$ is a multiple of 10, is
 \item $\overset{\rightarrow}{a}\;=\;\hat{i}+\alpha \hat{j}+3\hat{k}$ and $\overset{\rightarrow}{b}\;=\;3\hat{i}-\alpha \hat{j}+\hat{k}$. If the area of the parallelogram whose adjacent sides are represented by the vector $\overset{\rightarrow}{a}$ and $\overset{\rightarrow}{b}$ is $8\sqrt{3}$ square units, then $\overset{\rightarrow}{a}.\overset{\rightarrow}{b}$ is equal to
 \item If the curve $y=y\brak{x}$ represented by the solution of the differential equation $\brak{2xy^2-y}dx\;+\;xdy=0$, passes through the intersection of the lines, $2x-3y=1$ and $3x+2y=8$, then $|y\brak{1}|$ is equal to 
 \end{enumerate}
 
\end{document}

