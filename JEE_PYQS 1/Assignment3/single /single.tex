%iffalse
\let\negmedspace\undefined
\let\negthickspace\undefined
\documentclass[journal,,12pt,onecolumn]{IEEEtran}
\usepackage{cite}
\usepackage{amsmath,amssymb,amsfonts,amsthm}
\usepackage{algorithmic}
\usepackage{graphicx}
\usepackage{textcomp}
\usepackage{xcolor}
\usepackage{txfonts}
\usepackage{listings}
\usepackage{enumitem}
\usepackage{mathtools}
\usepackage{gensymb}
\usepackage{comment}
\usepackage[breaklinks=true]{hyperref}
\usepackage{tkz-euclide} 
\usepackage{listings}
\usepackage{gvv}                                        
%\def\inputGnumericTable{}                                 
\usepackage[latin1]{inputenc}                                
\usepackage{color}                                            
\usepackage{array}                                            
\usepackage{longtable}                                       
\usepackage{calc}                                             
\usepackage{multirow}                                         
\usepackage{hhline}                                           
\usepackage{ifthen}                                           
\usepackage{lscape}
\usepackage{tabularx}
\usepackage{array}
\usepackage{float}
\usepackage{multicol}



\newtheorem{theorem}{Theorem}[section]
\newtheorem{problem}{Problem}
\newtheorem{proposition}{Proposition}[section]
\newtheorem{lemma}{Lemma}[section]
\newtheorem{corollary}[theorem]{Corollary}
\newtheorem{example}{Example}[section]
\newtheorem{definition}[problem]{Definition}
\newcommand{\BEQA}{\begin{eqnarray}}
\newcommand{\EEQA}{\end{eqnarray}}
\newcommand{\define}{\stackrel{\triangle}{=}}
\theoremstyle{remark}
\newtheorem{rem}{Remark}

% Marks the beginning of the document
\begin{document}
\bibliographystyle{IEEEtran}
\vspace{3cm}

\title{25th February, 2021\\Shift-2}
\author{EE24BTECH11063 - Y.Harsha Vardhan Reddy}
\maketitle

\bigskip

\renewcommand{\thefigure}{\theenumi}
\renewcommand{\thetable}{\theenumi}

\section*{Single correct}
\begin{enumerate}
    \item Let A be a $3 \times 3$ matrix with det\brak{A} = 4. Let $R_i$ denote the $i^{th}$ row of A. If a matrix B is obtained by performing the operation $R_2 \rightarrow 2R_2 + 5R_3$ on 2A, then det\brak{B} is equal to: 
    \begin{enumerate}
    \begin{multicols}{4}
    \item 64
    \item 16
    \item 80
    \item 128
    \end{multicols}
        \end{enumerate}
        \item The shortest distance between the line $x-y=1$ and the curve $x^2=2y$ is :
        \begin{enumerate}
        \begin{multicols}{4}
            \item $\frac{1}{2}$
            \item 0
            \item $\frac{1}{2\sqrt{2}}$
            \item $\frac{1}{\sqrt{2}}$
            \end{multicols}
        \end{enumerate}
\item Let A be a set of all 4-digit natural numbers whose exactly one digit is 7. Then the probability that a randomly chosen element of A leaves remainder 2 when divided by 5 is:
        \begin{enumerate}
        \begin{multicols}{4}
        \item $\frac{1}{5}$
        \item $\frac{2}{9}$
        \item $\frac{97}{297}$
        \item $\frac{122}{297}$
        \end{multicols}
        \end{enumerate}
    \item $\cosec{\left[2\cot^{-1}{\brak{5}}+\cos^{-1}{\brak{\frac{4}{5}}}\right]}$ is equal to:
    \begin{enumerate}
        \begin{multicols}{4}
        \item $\frac{75}{56}$
        \item $\frac{65}{56}$
        \item $\frac{56}{33}$
        \item $\frac{65}{33}$
        \end{multicols}
        \end{enumerate}

 
 \item If $0 < x,y < \pi$ and $\cos{x}+\cos{y}-\cos{\brak{x+y}}=\frac{3}{2}$, then $\sin{x}+\cos{y}$ is equal to:
 \begin{enumerate}
     \begin{multicols}{4}
         \item $\frac{\brak{1+\sqrt{3}}}{2}$
         \item $\frac{\brak{1-\sqrt{3}}}{2}$
         \item $\frac{\sqrt{3}}{2}$
         \item $\frac{1}{2}$
     \end{multicols}
 \end{enumerate}
 \end{enumerate}

 
\end{document}

