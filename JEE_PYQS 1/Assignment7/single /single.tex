%iffalse
\let\negmedspace\undefined
\let\negthickspace\undefined
\documentclass[journal,,12pt,onecolumn]{IEEEtran}
\usepackage{cite}
\usepackage{amsmath,amssymb,amsfonts,amsthm}
\usepackage{algorithmic}
\usepackage{graphicx}
\usepackage{textcomp}
\usepackage{xcolor}
\usepackage{txfonts}
\usepackage{listings}
\usepackage{enumitem}
\usepackage{mathtools}
\usepackage{gensymb}
\usepackage{comment}
\usepackage[breaklinks=true]{hyperref}
\usepackage{tkz-euclide} 
\usepackage{listings}
\usepackage{gvv}                                        
%\def\inputGnumericTable{}                                 
\usepackage[latin1]{inputenc}                                
\usepackage{color}                                            
\usepackage{array}                                            
\usepackage{longtable}                                       
\usepackage{calc}                                             
\usepackage{multirow}                                         
\usepackage{hhline}                                           
\usepackage{ifthen}                                           
\usepackage{lscape}
\usepackage{tabularx}
\usepackage{array}
\usepackage{float}
\usepackage{multicol}



\newtheorem{theorem}{Theorem}[section]
\newtheorem{problem}{Problem}
\newtheorem{proposition}{Proposition}[section]
\newtheorem{lemma}{Lemma}[section]
\newtheorem{corollary}[theorem]{Corollary}
\newtheorem{example}{Example}[section]
\newtheorem{definition}[problem]{Definition}
\newcommand{\BEQA}{\begin{eqnarray}}
\newcommand{\EEQA}{\end{eqnarray}}
\newcommand{\define}{\stackrel{\triangle}{=}}
\theoremstyle{remark}
\newtheorem{rem}{Remark}

% Marks the beginning of the document
\begin{document}
\bibliographystyle{IEEEtran}
\vspace{3cm}

\title{31st January, 2024\\Shift-2}
\author{EE24BTECH11063 - Y.Harsha Vardhan Reddy}
\maketitle

\bigskip

\renewcommand{\thefigure}{\theenumi}
\renewcommand{\thetable}{\theenumi}

\section*{Single correct}
\begin{enumerate}
    \item Let $A\brak{a,b},\;B\brak{3,4}\text{ and } C\brak{-6,-8}$ respectively denote the centroid, circumcentre and orthocentre of a triangle. Then, the distance of the point $P\brak{2a+3,7b+5}$ from the line $2x+3y-4=0$ measured parallel to the line $x-2y-1=0$ is 
    \begin{enumerate}
        \begin{multicols}{4}
        \item $\frac{15\sqrt{5}}{7}$
            \item $\frac{\sqrt{5}}{17}$
            \item $\frac{17\sqrt{5}}{7}$
            \item $\frac{17\sqrt{5}}{6}$
        \end{multicols}
    \end{enumerate}
    \bigskip
\item The temperature $T\brak{t}$ of a body at a time $t=0$ is $160\degree$ F and it decreases continuously as per the differential equation $\frac{dT}{dt}=-K\brak{T-80}$, where $K$ is a positive constant. If $T\brak{15}=120\degree \;F$, then $T\brak{45}$ is equal to
\begin{enumerate}
    \begin{multicols}{4}
        \item $85\degree$ F
        \item $95\degree$ F
        \item $90\degree$ F
        \item $80\degree$ F
    \end{multicols}
\end{enumerate}
\bigskip
\item The area of the region enclosed by the parabolas $y=4x-x^2$ and $3y=\brak{x-4}^2$ is equal to
\begin{enumerate}
    \begin{multicols}{4}
        \item 6
        \item 4
        \item $\frac{32}{9}$
        \item $\frac{14}{3}$
    \end{multicols}
\end{enumerate}
\bigskip
\item The number of solutions, of the equation $e^{\sin{x}}-2e^{-\sin{x}}=2$, is:
\begin{enumerate}
    \begin{multicols}{2}
        \item 1
        \columnbreak
        \item 2
    \end{multicols}
    \begin{multicols}{2}
        \item more than 2
        \item 0
    \end{multicols}
\end{enumerate}
\bigskip
\item If for some m,n : ${}^{6}C_{m}+2\brak{{}^{6}C_{m+1}}+{}^{6}C_{m+2}\;>\; {}^{8}C_{3}$ and ${}^{n-1}P_{3}\;:\;{}^{n}P_{4}\;=\;1:8$, then ${}^{n}P_{m+1}+{}^{n+1}P_{m}$ is equal to
\begin{enumerate}
    \begin{multicols}{4}
        \item 380
        \item 384
        \item 376
        \item 372
    \end{multicols}
\end{enumerate}
 \end{enumerate}
 \bigskip

 
\end{document}

