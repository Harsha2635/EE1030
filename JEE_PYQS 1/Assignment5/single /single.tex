%iffalse
\let\negmedspace\undefined
\let\negthickspace\undefined
\documentclass[journal,,12pt,onecolumn]{IEEEtran}
\usepackage{cite}
\usepackage{amsmath,amssymb,amsfonts,amsthm}
\usepackage{algorithmic}
\usepackage{graphicx}
\usepackage{textcomp}
\usepackage{xcolor}
\usepackage{txfonts}
\usepackage{listings}
\usepackage{enumitem}
\usepackage{mathtools}
\usepackage{gensymb}
\usepackage{comment}
\usepackage[breaklinks=true]{hyperref}
\usepackage{tkz-euclide} 
\usepackage{listings}
\usepackage{gvv}                                        
%\def\inputGnumericTable{}                                 
\usepackage[latin1]{inputenc}                                
\usepackage{color}                                            
\usepackage{array}                                            
\usepackage{longtable}                                       
\usepackage{calc}                                             
\usepackage{multirow}                                         
\usepackage{hhline}                                           
\usepackage{ifthen}                                           
\usepackage{lscape}
\usepackage{tabularx}
\usepackage{array}
\usepackage{float}
\usepackage{multicol}



\newtheorem{theorem}{Theorem}[section]
\newtheorem{problem}{Problem}
\newtheorem{proposition}{Proposition}[section]
\newtheorem{lemma}{Lemma}[section]
\newtheorem{corollary}[theorem]{Corollary}
\newtheorem{example}{Example}[section]
\newtheorem{definition}[problem]{Definition}
\newcommand{\BEQA}{\begin{eqnarray}}
\newcommand{\EEQA}{\end{eqnarray}}
\newcommand{\define}{\stackrel{\triangle}{=}}
\theoremstyle{remark}
\newtheorem{rem}{Remark}

% Marks the beginning of the document
\begin{document}
\bibliographystyle{IEEEtran}
\vspace{3cm}

\title{28th June, 2022\\Shift-1}
\author{EE24BTECH11063 - Y.Harsha Vardhan Reddy}
\maketitle

\bigskip

\renewcommand{\thefigure}{\theenumi}
\renewcommand{\thetable}{\theenumi}

\section*{Single correct}
\begin{enumerate}
    \item The acute angle between the planes $P_1$ and $P_2$, when $P_1$ and $P_2$ are the planes passing through the intersection of the planes $5x+8y+13z-29=0$ and $8x-7y+z-20=0$ and the points $\brak{2,1,3}$ and $\brak{0,1,2}$, respectively, is
    \begin{enumerate}
    \begin{multicols}{4}
    \item $\frac{\pi}{3}$
    \item $\frac{\pi}{4}$
    \item $\frac{\pi}{6}$
    \item $\frac{\pi}{12}$
    \end{multicols}
        \end{enumerate}
        \bigskip
        \item Let the plane 
        \begin{align*}
        P\;:\;\overset{\rightarrow}{r}\cdot\overset{\rightarrow}{a}=d
        \end{align*}
        contain the line of intersection of two planes $\overset{\rightarrow}{r}\cdot \brak{\hat{i}+3\hat{j}-\hat{k}}=6$ and $\overset{\rightarrow}{r}\cdot \brak{-6\hat{i}+5\hat{j}-\hat{k}}=7$. If the plane $P$ passes through the point $\brak{2,3,\frac{1}{2}}$, then the value of $\frac{|13\overset{\rightarrow}{a}|^2}{d^2}$ is equal to
        \begin{enumerate}
        \begin{multicols}{4}
            \item 90
            \item 93
            \item 95
            \item 97
            \end{multicols}
        \end{enumerate}
        \bigskip
\item The probability, that in a randomly selected 3-digit number at least two digits are odd, is
        \begin{enumerate}
        \begin{multicols}{4}
        \item $\frac{19}{36}$
        \item $\frac{15}{36}$
        \item $\frac{13}{36}$
        \item $\frac{23}{36}$
        \end{multicols}
        \end{enumerate}
        \bigskip
    \item Let $AB$ and $PQ$ be two vertical poles, $160m$ apart from each other. Let $C$ be the middle point of $B$ and $Q$, which are feet of these two poles. Let $\frac{\pi}{8}$ and $\theta$ be the angles of elevation from $C$ to $P$ and $A$, respectively. If the height of pole $AB$, then $\tan^2{\theta}$ is equal to
    \begin{enumerate}
        \begin{multicols}{4}
        \item $\frac{3-2\sqrt{2}}{2}$
        \item $\frac{3+\sqrt{2}}{2}$
        \item $\frac{3-2\sqrt{2}}{4}$
        \item $\frac{3-\sqrt{2}}{4}$
        \end{multicols}
        \end{enumerate}
\bigskip
\item Let $p,q,r$ be three logical statements. Consider the compound statements
\begin{align*}
S_1 \;:\; ((\sim p) \lor q) \lor ((\sim p) \lor r) \ \text{ and } \\
S_2 \;:\; p \rightarrow (q \lor r)
\end{align*}
 Then, which of the following is NOT true?
 \begin{enumerate}
 \item If $S_2$ is True, then $S_1$ is True
 \item If $S_2$ is False, then $S_1$ is False
 \item If $S_2$ is False, then $S_1$ is True
\item If $S_1$ is False, then $S_2$ is False
 \end{enumerate}
 \end{enumerate}

 
\end{document}

