%iffalse
\let\negmedspace\undefined
\let\negthickspace\undefined
\documentclass[journal,,12pt,onecolumn]{IEEEtran}
\usepackage{cite}
\usepackage{amsmath,amssymb,amsfonts,amsthm}
\usepackage{algorithmic}
\usepackage{graphicx}
\usepackage{textcomp}
\usepackage{xcolor}
\usepackage{txfonts}
\usepackage{listings}
\usepackage{enumitem}
\usepackage{mathtools}
\usepackage{gensymb}
\usepackage{comment}
\usepackage[breaklinks=true]{hyperref}
\usepackage{tkz-euclide} 
\usepackage{listings}
\usepackage{gvv}                                        
%\def\inputGnumericTable{}                                 
\usepackage[latin1]{inputenc}                                
\usepackage{color}                                            
\usepackage{array}                                            
\usepackage{longtable}                                       
\usepackage{calc}                                             
\usepackage{multirow}                                         
\usepackage{hhline}                                           
\usepackage{ifthen}                                           
\usepackage{lscape}
\usepackage{tabularx}
\usepackage{array}
\usepackage{float}
\usepackage{multicol}



\newtheorem{theorem}{Theorem}[section]
\newtheorem{problem}{Problem}
\newtheorem{proposition}{Proposition}[section]
\newtheorem{lemma}{Lemma}[section]
\newtheorem{corollary}[theorem]{Corollary}
\newtheorem{example}{Example}[section]
\newtheorem{definition}[problem]{Definition}
\newcommand{\BEQA}{\begin{eqnarray}}
\newcommand{\EEQA}{\end{eqnarray}}
\newcommand{\define}{\stackrel{\triangle}{=}}
\theoremstyle{remark}
\newtheorem{rem}{Remark}

% Marks the beginning of the document
\begin{document}
\bibliographystyle{IEEEtran}
\vspace{3cm}

\title{25th February, 2021\\Shift-2}
\author{EE24BTECH11063 - Y.Harsha Vardhan Reddy}
\maketitle

\bigskip

\renewcommand{\thefigure}{\theenumi}
\renewcommand{\thetable}{\theenumi}


\section*{Integer Type}
\begin{enumerate}
 \item Let $R_1$ and $R_2$ be relations on the set $\{1,2,\cdots , 50\}$ such that $R_1\;=\;\{\brak{p,p^n}:p\text{ is a prime and }n\ge 0 \text{is an integer}\}$ and $R_2\;=\;\{\brak{p,p^n}:p\text{ is a prime and }n=0 \text{ or } 1\}$. Then, the number of elements in $R_1-R_2$ is
 \bigskip
 \item The number of real solutions of the equation $e^{4x}+4e^{3x}-58e^{2x}+4e^{x}+1=0$ is
 \bigskip
 \item The mean and standard deviation of 15 observations are found to be 8 and 3, respectively. On rechecking, it was found that, in the observations, 20 was misread as 5. Then, the correct value of variance is equal to
 \bigskip
 \item If
 \begin{align*}
 \overset{\rightarrow}{a}=2\hat{i}+\hat{j}+3\hat{k},\overset{\rightarrow}{b}=3\hat{i}+3\hat{j}+\hat{k}\text{ and }\overset{\rightarrow}{c}=c_1\hat{i}+c_2\hat{j}+c_3\hat{k}
 \end{align*}
 are coplanar vectors and 
 \begin{align*}
 \overset{\rightarrow}{a} \cdot \overset{\rightarrow}{c}=5, \overset{\rightarrow}{b}\perp \overset{\rightarrow}{c}
 \end{align*}
 then $122\brak{c_1+c_2+c_3}$ is equal to
 \bigskip
 \item A ray of light passing through the point $P \; \brak{2,3}$ reflects on the x-axis at point $A$ and the reflected ray passes through the point $Q\;\brak{5,4}$. Let $R$ be the point that divides the line segment $AQ$ internally into the ratio $2\;:\;1$. Let the co-ordinates of the foot of the perpendicular $M$ from $R$ on the bisector of the angle $PAQ$ be $\brak{\alpha,\beta}$. Then, the value of $7\alpha\;+\;3\beta$ is equal to
 \bigskip
 \item Let $l$ be a line which is normal to the curve $y=2x^2+x+2$ at a point $P$ on the curve. If the point $Q\;\brak{6,4}$ lies on the line $l$ and $O$ is the origin, then the area of the triangle $OPQ$ is equal to
 \bigskip
 \item Let $A\;=\;{1,a_1,a_2,\cdots a_{18},77}$ be a set of integers with $1\;<\;a_1\;<\;a_2\;<\cdots<\;a_{18}\;<\;77$. Let the set $A\;A\;={x+y:x,y \in A}$ contain exactly 39 elements. Then, the value of $a_1\;+\;a_2\;+\;\cdots+\;a_{18}$ is equal to
 \bigskip
 \item The number of positive integers $k$ such that the constant term in the binomial expansion of
 \begin{align*}
 \brak{2x^3+\frac{3}{x^k}}^{12},\;x\neq0
 \end{align*}
 is $2^8\cdot I$, where $I$ is an odd integer, is
 \bigskip
 \item The number of elements in the set 
 \begin{align*}
 \{z\;=\;a+ib \in C\; : \; a,b\;\in Z\;\text{ and } 1<|z-3+2i|<4\}
 \end{align*}
 is
 \bigskip
 \item Let the lines 
 \begin{align*}
 y\;+\;2x=\sqrt{11}+7\sqrt{7}\text{ and } 2y+x\;=\;2\sqrt{11}+6\sqrt{7}
 \end{align*}
 be normal to a circle $C\; :\;\brak{x-h}^2\;+\;\brak{y-k}^2=r^2$. If the line $\sqrt{11}y-3x=\frac{5\sqrt{17}}{3}+11$ is tangent to the circle $C$, then the value of $\brak{5h-8k}^2+5r^2$ is equal to
 \end{enumerate}
 
\end{document}

