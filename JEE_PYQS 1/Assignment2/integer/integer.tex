%iffalse
\let\negmedspace\undefined
\let\negthickspace\undefined
\documentclass[journal,,12pt,onecolumn]{IEEEtran}
\usepackage{cite}
\usepackage{amsmath,amssymb,amsfonts,amsthm}
\usepackage{algorithmic}
\usepackage{graphicx}
\usepackage{textcomp}
\usepackage{xcolor}
\usepackage{txfonts}
\usepackage{listings}
\usepackage{enumitem}
\usepackage{mathtools}
\usepackage{gensymb}
\usepackage{comment}
\usepackage[breaklinks=true]{hyperref}
\usepackage{tkz-euclide} 
\usepackage{listings}
\usepackage{gvv}                                        
%\def\inputGnumericTable{}                                 
\usepackage[latin1]{inputenc}                                
\usepackage{color}                                            
\usepackage{array}                                            
\usepackage{longtable}                                       
\usepackage{calc}                                             
\usepackage{multirow}                                         
\usepackage{hhline}                                           
\usepackage{ifthen}                                           
\usepackage{lscape}
\usepackage{tabularx}
\usepackage{array}
\usepackage{float}
\usepackage{multicol}



\newtheorem{theorem}{Theorem}[section]
\newtheorem{problem}{Problem}
\newtheorem{proposition}{Proposition}[section]
\newtheorem{lemma}{Lemma}[section]
\newtheorem{corollary}[theorem]{Corollary}
\newtheorem{example}{Example}[section]
\newtheorem{definition}[problem]{Definition}
\newcommand{\BEQA}{\begin{eqnarray}}
\newcommand{\EEQA}{\end{eqnarray}}
\newcommand{\define}{\stackrel{\triangle}{=}}
\theoremstyle{remark}
\newtheorem{rem}{Remark}

% Marks the beginning of the document
\begin{document}
\bibliographystyle{IEEEtran}
\vspace{3cm}

\title{6th September, 2020\\Shift-2}
\author{EE24BTECH11063 - Y.Harsha Vardhan Reddy}
\maketitle

\bigskip

\renewcommand{\thefigure}{\theenumi}
\renewcommand{\thetable}{\theenumi}


\section*{Integer Type}
\begin{enumerate}
 \item The number of words(with or without meaning) that can be formed from all the letters of the word "LETTER" in which vowels never come together is:
 \item If $\Bar{x}$ and $\Bar{y}$ be two non-zero vectors such that $|\Bar{x}+\Bar{y}|=|\Bar{x}|$ and $2\Bar{x}+\lambda \Bar{y}$ is perpendicular to $\Bar{y}$, then the value of $\lambda$ is
 \item Consider the data on x taking the values 0,2,4,8, ...,2n with frequencies ${}^{n}C_{0}
,{}^{n}C_{1},{}^{n}C_{2}, ...,{}^{n}C_{n}$,respectively. If the mean of this data is $\frac{728}{2^n}$, then n is equal to:
 \item Suppose that function $f\;:\;R\rightarrow R$ satisfies $f\brak{x+y}=f\brak{x}f\brak{y}$ for all $x,y \in R$ and $f\brak{1}=3$. If $\sum_{i=1}^{n} f(i)\;=\;363$, then n is equal to:

 \item The sum of distinct values of $\lambda$ for which the system of equations 
 \begin{align*}
     \brak{\lambda -1}x +\brak{3\lambda+1}y+2\lambda =0
 \end{align*}
 \begin{align*}
     \brak{\lambda -1}x +\brak{4\lambda-2}y+\brak{\lambda+3}z =0
 \end{align*}
 \begin{align*}
     2x +\brak{3\lambda+1}y+3\brak{\lambda-1}z =0
 \end{align*}
 has non-zero solutions, is :
 \end{enumerate}
 
\end{document}

