%iffalse
\let\negmedspace\undefined
\let\negthickspace\undefined
\documentclass[journal,,12pt,onecolumn]{IEEEtran}
\usepackage{cite}
\usepackage{amsmath,amssymb,amsfonts,amsthm}
\usepackage{algorithmic}
\usepackage{graphicx}
\usepackage{textcomp}
\usepackage{xcolor}
\usepackage{txfonts}
\usepackage{listings}
\usepackage{enumitem}
\usepackage{mathtools}
\usepackage{gensymb}
\usepackage{comment}
\usepackage[breaklinks=true]{hyperref}
\usepackage{tkz-euclide} 
\usepackage{listings}
\usepackage{gvv}                                        
%\def\inputGnumericTable{}                                 
\usepackage[latin1]{inputenc}                                
\usepackage{color}                                            
\usepackage{array}                                            
\usepackage{longtable}                                       
\usepackage{calc}                                             
\usepackage{multirow}                                         
\usepackage{hhline}                                           
\usepackage{ifthen}                                           
\usepackage{lscape}
\usepackage{tabularx}
\usepackage{array}
\usepackage{float}
\usepackage{multicol}



\newtheorem{theorem}{Theorem}[section]
\newtheorem{problem}{Problem}
\newtheorem{proposition}{Proposition}[section]
\newtheorem{lemma}{Lemma}[section]
\newtheorem{corollary}[theorem]{Corollary}
\newtheorem{example}{Example}[section]
\newtheorem{definition}[problem]{Definition}
\newcommand{\BEQA}{\begin{eqnarray}}
\newcommand{\EEQA}{\end{eqnarray}}
\newcommand{\define}{\stackrel{\triangle}{=}}
\theoremstyle{remark}
\newtheorem{rem}{Remark}

% Marks the beginning of the document
\begin{document}
\bibliographystyle{IEEEtran}
\vspace{3cm}

\title{9th September, 2020\\Shift-2}
\author{EE24BTECH11063 - Y.Harsha Vardhan Reddy}
\maketitle

\bigskip

\renewcommand{\thefigure}{\theenumi}
\renewcommand{\thetable}{\theenumi}

\section*{Single correct}
\begin{enumerate}
    \item If for some $\alpha\in\;R$, the lines $L_1 : \frac{\brak{x+1}}{2}=\frac{\brak{y-2}}{-1}=\frac{\brak{z-1}}{2}$ and $L_2 : \frac{\brak{x+2}}{\alpha}=\frac{\brak{y+1}}{5-\alpha}=\frac{\brak{z+1}}{1}$ are coplanar, the the line $L_2$ passes through the point:
    \begin{enumerate}
    
        
    
        \item \brak{2, -10. -2}
        \item \brak{10, -2, -2}
        \item \brak{10, 2, 2}
        \item \brak{-2, 10, 2}
        
        \end{enumerate}
        \item The value of $\left[\frac{\brak{-1+i\sqrt{3}}}{\brak{1-i}}\right]^{30}$
        \begin{enumerate}
        \begin{multicols}{4}
            \item $2^{15}i$
            \item $-2^{15}$
            \item $-2^{15}i$
            \item $6^5$
            \end{multicols}
        \end{enumerate}
\item Let y=y\brak{x} be the solution of the differential equation $\cos{x}\brak{\frac{dy}{dx}}+2y \sin{x}\;=\;\sin{2x},\;x\in\brak{0,\frac{\pi}{2}}$. If $y\brak{\frac{\pi}{3}}=0$, then $y\brak{\frac{\pi}{4}}$ is equal to:
        \begin{enumerate}
        \item $2+\sqrt{2}$
        \item $\sqrt{2}-2$
        \item $\brak{\frac{1}{\sqrt{2}}}-1$
        \item $2-\sqrt{2}$
        \end{enumerate}
    \item If the system of linear equations 
    \begin{align*}
        x+y+3z=0
    \end{align*}
    \begin{align*}
        x+3y+k^2z=0
    \end{align*}
    \begin{align*}
        3x+y+3z=0
    \end{align*}
    has a non-zero solution \brak{x,y,z} for some $k\in R$, then x+\brak{\frac{y}{z}} is equal to :
    \begin{enumerate}
    
        \item -9
        \item 9
        \item -3
        \item 3
       \end{enumerate}

 
 \item Which of the following points lies on the tangent to the curve $4x^3e^y+x^4e^y+2\sqrt{y+1}=3$ at the point \brak{1,0} ?
 \begin{enumerate}
     \item \brak{2,6}
     \item \brak{2,2}
     \item \brak{-2,6}
     \item \brak{-2,4}
 \end{enumerate}
 \end{enumerate}
\section*{Integer Type}
\begin{enumerate}
 \item Let A = \{a,b,c\} and B=\{1,2,3,4\}. Then the number of elements in the set C=\{f : $A \rightarrow B$ $2 \in f\brak{A}$ and f is not one-one\} is :
 \item The coefficient of $x^4$ in the expansion of $\brak{1+x+x^2+x^3}^6$ in powers of x, is:
 \item Let the vectors $\Bar{a},\Bar{b},\Bar{c}$ such that $|\Bar{a}|=2,|\Bar{b}|=4$ and $|\Bar{c}|=4$. If the projection of vector b on vector a is equal to the projection of vector c on vector a and b is perpendicular to vector c, then the value $|\Bar{a}+\Bar{b}-\Bar{c}|$ is:
 \item If the lines $x+y=a$ and $x-y=b$ touch the curve $y=x^2-3x+2$ at the points where the curve intersects the x-axis, then a/b is equal to:
 \item In a bombing attack, there is 50\% chance that a bomb will hit the target.At least two independent hits are required to destroy the target completely. Then the minimum number of bombs, that must be dropped to ensure that there is at least 99\% chance of completely destroying the target, is
 \end{enumerate}
 
\end{document}

