%iffalse
\let\negmedspace\undefined
\let\negthickspace\undefined
\documentclass[journal,,12pt,twocolumn]{IEEEtran}
\usepackage{cite}
\usepackage{amsmath,amssymb,amsfonts,amsthm}
\usepackage{algorithmic}
\usepackage{graphicx}
\usepackage{textcomp}
\usepackage{xcolor}
\usepackage{txfonts}
\usepackage{listings}
\usepackage{enumitem}
\usepackage{mathtools}
\usepackage{gensymb}
\usepackage{comment}
\usepackage[breaklinks=true]{hyperref}
\usepackage{tkz-euclide} 
\usepackage{listings}
\usepackage{gvv}                                        
%\def\inputGnumericTable{}                                 
\usepackage[latin1]{inputenc}                                
\usepackage{color}                                            
\usepackage{array}                                            
\usepackage{longtable}                                       
\usepackage{calc}                                             
\usepackage{multirow}                                         
\usepackage{hhline}                                           
\usepackage{ifthen}                                           
\usepackage{lscape}
\usepackage{tabularx}
\usepackage{array}
\usepackage{float}
\usepackage{multicol}



\newtheorem{theorem}{Theorem}[section]
\newtheorem{problem}{Problem}
\newtheorem{proposition}{Proposition}[section]
\newtheorem{lemma}{Lemma}[section]
\newtheorem{corollary}[theorem]{Corollary}
\newtheorem{example}{Example}[section]
\newtheorem{definition}[problem]{Definition}
\newcommand{\BEQA}{\begin{eqnarray}}
\newcommand{\EEQA}{\end{eqnarray}}
\newcommand{\define}{\stackrel{\triangle}{=}}
\theoremstyle{remark}
\newtheorem{rem}{Remark}

% Marks the beginning of the document
\begin{document}
\bibliographystyle{IEEEtran}
\vspace{3cm}

\title{Chapter 16\\Application of derivatives}
\author{EE24BTECH11063 - Y.Harsha Vardhan Reddy}
\maketitle
\newpage
\bigskip

\renewcommand{\thefigure}{\theenumi}
\renewcommand{\thetable}{\theenumi}

\section*{G : Comprehension Based Questions}
\section*{PASSAGE-1}
If a continuous function $f$  defined on a real line $R$, assumes positive and negative values in $R$ then the equation $f(x)=0$ has a root in $R$. For example, if it is known that a continuous function $f$ on $R$ is positive at some point and its minimum value is negative then the equation $f(x)=0$ has a root in $R$.\\
Consider $f(x)=ke^x-x$ for all real $x$ where $k$ is a real constant.
\begin{enumerate}
\item The line $y=x$ meets $y=ke^x$ for $k \le 0$ at
    \begin{enumerate}
\begin{multicols}{2}
\item no point
\columnbreak
\item one point
\end{multicols}
\begin{multicols}{2}
\item two points
\item more than two points
\end{multicols}
\end{enumerate}

  
\hfill {(2007-4marks)}\\


\item  The positive value of $k$ for which $ke^x-x=0$ has only one root is
\begin{enumerate}
\begin{multicols}{2}
\item $\frac{1}{e}$
\columnbreak
\item 1
\end{multicols}
\begin{multicols}{2}
\item $e$
\item $\log_e{2}$
\end{multicols}
\end{enumerate}

\hfill {(2007-4marks)}


\item For $k>0$, the set of all values of $k$ for which $ke^x-x=0$ has two distinct roots is 
\begin{enumerate}
    \begin{multicols}{2}
    \item $\brak{0,\frac{1}{e}}$
    \columnbreak
    \item $\brak{\frac{1}{e},1}$
    \end{multicols}
    \begin{multicols}{2}
    \item $\brak{\frac{1}{e},\infty}$
    \item $\brak{0,1}$
    \end{multicols}
\end{enumerate}

\hfill {(2007-4marks)}


\section*{PASSAGE-2}
Let $f(x)=\brak{1-x}^2 \sin^2 x + x^2$ for all $x \in \mathbb{IR}$ and let $g(x)=
\int_{1}^{x} \brak{\frac{2(t-1)}{t+1} - \ln t}  f(t) \, dt $ for all $x \in (1 ,\infty)$.
\item Consider the statements:\\
P : There exists some $x \in \mathbb{R}$ such that $f(x) + 2x = 2\brak{1+x^2}$\\
Q : There exists some $x \in\mathbb{R}$ such that $2f(x) + 1 = 2x\brak{1+x}$\\
    Then
\begin{enumerate}
\begin{multicols}{1}
    

\item both $P$ and $Q$ are true
\item $P$ is true and $Q$ is false
\item $P$ is false and $Q$ is true
\item both $P$ and $Q$ are true
\end{multicols}
\end{enumerate}

\hfill{(2012)}



\item Which of the following is true?
\begin{enumerate}
\begin{multicols}{1}
\item $g$ is increasing on $\brak{1,\infty}$
\item $g$ is decreasing on $\brak{1,\infty}$
\item $g$ is increasing in (1,2) and decreasing on $\brak{2,\infty}$
\item $g$ is decreasing in (1,2) and increasing on $\brak{2,\infty}$
\end{multicols}
\end{enumerate}
\hfill{(2012)}

\section*{PASSAGE-3}
Let $f(x) : [0,1] \to\mathbb{R}$
(the set of all real numbers) be a function. Suppose the function $f$ is twice differentiable , $f(0)=f(1)=0$ and satisfies \begin{align*} f''(x)-2f'(x)+f(x) \geq e^x , x \in [0,1].\end{align*} 

\item Which of the following is true for $0<x<1$?


\hfill{(JEE Adv. 2013)}

\begin{enumerate}
\begin{multicols}{2}
\item $0<f(x)< \infty$ 
\columnbreak
\item $ -\frac{1}{2} <f(x)< \frac{1}{2}$
\end{multicols}
\begin{multicols}{2}
\item $-\frac{1}{4}<f(x)<1$
\item $-\infty <f(x)<0$
\end{multicols}
\end{enumerate}


\item If the function $e^{-x}f(x)$ assumes its minimum in the interval [0,1] at $x=\frac{1}{4}$, which of the following is true?
\end{enumerate}

\hfill{(JEE Adv. 2013)}

\begin{enumerate}
\begin{multicols}{1}
\item $f'(x)<f(x)$ , $\frac{1}{4}<x<\frac{3}{4}$ \\

\item $f'(x)>f(x)$ , $0<x<\frac{1}{4}$ \\ 

\item $f'(x)<f(x)$ , $0<x<\frac{1}{4}$ \\

\item $f'(x)<f(x)$ , $\frac{3}{4}<x<1$ \\
\end{multicols}

\end{enumerate}


\section*{I:Integer Value Correct Type}

\begin{enumerate}
\item The maximum value of the function \\
$f(x)=2x^3-15x^2+36x-48$ on the set\\
$A=\cbrak{x|x^2+20 \le 9x}$ is

\hfill {(2009)}

\item Let $p(x)$ be a polynomial of degree 4 having extremum at $x=1,2 $ and $\lim \limits_{x\to 0} \brak{1+\frac{p(x)}{x^2}} = 2$.\\
Then the value of $p(2)$ is
\hfill{(2009)}
\item Let $f$ be a real-valued differentiable function on $\textbf{R}$(the set of all real numbers) such that $f(1)=1$. If the y-intercept of the tangent at any point $P(x,y)$ on the curve $y=f(x)$ is equal to the cube of the abscissa of $P$, then find the value of $f(-3)$

    \hfill {(2010)}


\end{enumerate}
\end{document}
