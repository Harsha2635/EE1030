\let\negmedspace\undefined
\let\negthickspace\undefined
\documentclass[journal]{IEEEtran}
\usepackage[a5paper, margin=10mm, onecolumn]{geometry}
%\usepackage{lmodern} % Ensure lmodern is loaded for pdflatex
\usepackage{tfrupee} % Include tfrupee package

\setlength{\headheight}{1cm} % Set the height of the header box
\setlength{\headsep}{0mm}     % Set the distance between the header box and the top of the text

\usepackage{gvv-book}
\usepackage{gvv}
\usepackage{cite}
\usepackage{amsmath,amssymb,amsfonts,amsthm}
\usepackage{algorithmic}
\usepackage{graphicx}
\usepackage{textcomp}
\usepackage{xcolor}
\usepackage{txfonts}
\usepackage{listings}
\usepackage{enumitem}
\usepackage{mathtools}
\usepackage{gensymb}
\usepackage{comment}
\usepackage[breaklinks=true]{hyperref}
\usepackage{tkz-euclide} 
\usepackage{listings}
% \usepackage{gvv}                                        
\def\inputGnumericTable{}                                 
\usepackage[latin1]{inputenc}                                
\usepackage{color}                                            
\usepackage{array}                                            
\usepackage{longtable}                                       
\usepackage{calc}                                             
\usepackage{multirow}                                         
\usepackage{hhline}                                           
\usepackage{ifthen}                                           
\usepackage{lscape}

\begin{document}

\bibliographystyle{IEEEtran}
\vspace{3cm}

\title{1-1.2-24}
\author{EE24BTECH11063 - Y.Harsha Vardhan Reddy
}
% \maketitle
% \newpage
% \bigskip
{\let\newpage\relax\maketitle}

\renewcommand{\thefigure}{\theenumi}
\renewcommand{\thetable}{\theenumi}
\setlength{\intextsep}{10pt} % Space between text and floats


\numberwithin{equation}{enumi}
\numberwithin{figure}{enumi}
\renewcommand{\thetable}{\theenumi}
\textbf{Question}:\\
A man goes 5 meters due west and then 12 meters due north. How far is he from the starting point?
\\
\solution
\begin{table}[h!]    
  \centering
  \begin{tabular}[12pt]{ |c| c|}
    \hline
    \textbf{Variable} & \textbf{Description}\\ 
    \hline
    $V,u,f$ & Parameters of conic \\
    \hline 
    $h,m$ & Parameters of line \\
    \hline
     $c$ & length of side-AB \\
     \hline
     $P_1,P_2$ & Points of intersection \\
    \hline
\end{tabular}

  \caption{Variables Used}
  \label{tab1-1.2-20}
\end{table}
\begin{align}
x_1=\myvec{-5\\0} \\
x_2=\myvec{0\\12}
\end{align}
Net displacement is given by,
\begin{align}
x=x_1 + x_2\\
x=\myvec{-5\\0}-\myvec{0\\12}=\myvec{-5\\12}\\
    x^T=\myvec{-5 & 12}
\end{align}
Therefore, the magnitude of x is given by,
\begin{align}
    ||x||=\sqrt{x^T * x}\\
    ||x||=\sqrt{25+144}=\sqrt{169}=13 units
\end{align}
Therefore, the distance of man from starting point is 13 units.
\begin{figure}[h!]
   \centering
   \includegraphics[width=0.7\linewidth]{figure_1.png}
   \label{stemplot}
\end{figure}





\end{document}
