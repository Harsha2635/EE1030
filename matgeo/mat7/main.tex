\let\negmedspace\undefined
\let\negthickspace\undefined
\documentclass[journal]{IEEEtran}
\usepackage[a5paper, margin=10mm, onecolumn]{geometry}
%\usepackage{lmodern} % Ensure lmodern is loaded for pdflatex
\usepackage{tfrupee} % Include tfrupee package

\setlength{\headheight}{1cm} % Set the height of the header box
\setlength{\headsep}{0mm}     % Set the distance between the header box and the top of the text

\usepackage{gvv-book}
\usepackage{gvv}
\usepackage{cite}
\usepackage{amsmath,amssymb,amsfonts,amsthm}
\usepackage{algorithmic}
\usepackage{graphicx}
\usepackage{textcomp}
\usepackage{xcolor}
\usepackage{txfonts}
\usepackage{listings}
\usepackage{enumitem}
\usepackage{mathtools}
\usepackage{gensymb}
\usepackage{comment}
\usepackage[breaklinks=true]{hyperref}
\usepackage{tkz-euclide} 
\usepackage{listings}
% \usepackage{gvv}                                        
\def\inputGnumericTable{}                                 
\usepackage[latin1]{inputenc}                                
\usepackage{color}                                            
\usepackage{array}                                            
\usepackage{longtable}                                       
\usepackage{calc}                                             
\usepackage{multirow}                                         
\usepackage{hhline}                                           
\usepackage{ifthen}                                           
\usepackage{lscape}

\begin{document}

\bibliographystyle{IEEEtran}
\vspace{3cm}

\title{9-9.5-1}
\author{EE24BTECH11063 - Y.Harsha Vardhan Reddy}
% \maketitle
% \newpage
% \bigskip
{\let\newpage\relax\maketitle}

\renewcommand{\thefigure}{\theenumi}
\renewcommand{\thetable}{\theenumi}
\setlength{\intextsep}{10pt} % Space between text and floats


\numberwithin{equation}{enumi}
\numberwithin{figure}{enumi}
\renewcommand{\thetable}{\theenumi}
\textbf{Question}:\\
Find the area of the region 

\[
\{(x,y) : x^2 + y^2 \leq 16a^2 \text{ and } y^2 \leq 6ax\}
\]
\\
\solution
\begin{table}[h!]    
  \centering
  \begin{tabular}[12pt]{ |c| c|}
    \hline
    \textbf{Variable} & \textbf{Description}\\ 
    \hline
    $V,u,f$ & Parameters of conic \\
    \hline 
    $h,m$ & Parameters of line \\
    \hline
     $c$ & length of side-AB \\
     \hline
     $P_1,P_2$ & Points of intersection \\
    \hline
\end{tabular}

  \caption{Variables Used}
  \label{tab1-1.2-20}
\end{table}
The parameters of the conics are
\begin{align}
V_1=\myvec{0 & 0 \\ 0 & 1}\;,\;u_1=\myvec{3a\\0} \;,\;f_1=0\\
V_2=\myvec{1 & 0 \\ 0 & 1}\;,\;u_2=\myvec{0\\0}\;,\;f_2=-16a^2
\end{align}
The intersection of two conics with parameters $V_i,u_i,f_i,\;i= 1,2$ is defined as
\begin{align}
x^T\brak{V_1+\mu V_2}x+2\brak{u_1+\mu u_2}^T x + \brak{f_1+\mu f_2}\;=\;0
\end{align}
Solving this the points of intersection are
\begin{align}
\myvec{2a\\\sqrt{12}a}\;,\;\myvec{2a\\-\sqrt{12}a}
\end{align}
Area between the curves is,
\begin{align}
\int_{0}^{2a} \brak{\sqrt{16a^2-x^2}-\sqrt{6ax}} \, dx \\
=\brak{2\sqrt{3}-\frac{8\sqrt{6}}{3}+\frac{4\pi}{3}}
\end{align}
By solving the integration, we get area is equal to $1.10a^2$ sq.units
\begin{figure}[h!]
   \centering
   \includegraphics[width=\linewidth]{figs/figure_1.png}
   \label{stemplot}
   \caption{}
\end{figure}





\end{document}
