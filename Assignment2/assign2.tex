%iffalse
\let\negmedspace\undefined
\let\negthickspace\undefined
\documentclass[journal,,12pt,twocolumn]{IEEEtran}
\usepackage{cite}
\usepackage{amsmath,amssymb,amsfonts,amsthm}
\usepackage{algorithmic}
\usepackage{graphicx}
\usepackage{textcomp}
\usepackage{xcolor}
\usepackage{txfonts}
\usepackage{listings}
\usepackage{enumitem}
\usepackage{mathtools}
\usepackage{gensymb}
\usepackage{comment}
\usepackage[breaklinks=true]{hyperref}
\usepackage{tkz-euclide} 
\usepackage{listings}
\usepackage{gvv}                                        
%\def\inputGnumericTable{}                                 
\usepackage[latin1]{inputenc}                                
\usepackage{color}                                            
\usepackage{array}                                            
\usepackage{longtable}                                       
\usepackage{calc}                                             
\usepackage{multirow}                                         
\usepackage{hhline}                                           
\usepackage{ifthen}                                           
\usepackage{lscape}
\usepackage{tabularx}
\usepackage{array}
\usepackage{float}
\usepackage{multicol}


\newtheorem{theorem}{Theorem}[section]
\newtheorem{problem}{Problem}
\newtheorem{proposition}{Proposition}[section]
\newtheorem{lemma}{Lemma}[section]
\newtheorem{corollary}[theorem]{Corollary}
\newtheorem{example}{Example}[section]
\newtheorem{definition}[problem]{Definition}
\newcommand{\BEQA}{\begin{eqnarray}}
\newcommand{\EEQA}{\end{eqnarray}}
\newcommand{\define}{\stackrel{\triangle}{=}}
\theoremstyle{remark}
\newtheorem{rem}{Remark}

% Marks the beginning of the document
\begin{document}

\bibliographystyle{IEEEtran}
\vspace{3cm}

\title{Assignment-2}
\author{EE24BTECH11063 - Y.Harsha Vardhan Reddy}
\maketitle
\newpage
\bigskip

\renewcommand{\thefigure}{\theenumi}
\renewcommand{\thetable}{\theenumi}
\section*{Chapter 15\\Matrices and Determinants}
\section*{Single Correct Type}
\begin{enumerate}
    \item Let k be an integer such that triangle with vertices \brak{k,-3k}, \brak{5,k} and \brak{-k,2} has area 28 sq. units. Then the orthocentre of the triangle is at the point :
    \hfill{[JEE M 2017]}
    \begin{enumerate}
    \begin{multicols}{2}
        \item \brak{2,\frac{1}{2}}
        \columnbreak
        \item \brak{2,-\frac{1}{2}}
        \end{multicols}
        \begin{multicols}{2}
        \item \brak{1,\frac{3}{4}}
        \item \brak{1,-\frac{3}{4}}
        \end{multicols}
    \end{enumerate}
    \item Let $\omega$ be a complex number such that $2\omega + 1=z$ where $z=\sqrt{-3}$. If $\begin{vmatrix}
1 & 1  & 1 \\
1 & -\omega^2-1 & \omega^2 \\
1 & \omega^2 & \omega^7 
\end{vmatrix}$ $=3k$, then k is equal to:
\hfill{[JEE M 2017]}
\begin{enumerate}
\begin{multicols}{2}
    \item 1 
    \columnbreak
    \item -z
    \end{multicols}
    \begin{multicols}{2}
    \item z
    \item -1
    \end{multicols}
\end{enumerate}
\item If A= $\begin{bmatrix}
    2 & -3 \\
    -4 & 1
\end{bmatrix}$
    , then $adj(3A^2+12A)$ is equal to:
\hfill{[JEE M 2017]}
\begin{enumerate}
\begin{multicols}{2}
    \item $\begin{bmatrix}
    72 & -63 \\
    -84 & 51
\end{bmatrix}$ 
\columnbreak

    \item $\begin{bmatrix}
    72 & -84 \\
    -63 & 51 
\end{bmatrix}$ \\
\end{multicols}
\begin{multicols}{2}
    \item $\begin{bmatrix}
    51 & 63 \\
    84 & 72
\end{bmatrix}$ \\

    \item $\begin{bmatrix}
    51 & 84 \\
    63 & 72
    
\end{bmatrix}$
\end{multicols}
\end{enumerate} 
\item If $\begin{vmatrix}
x-4 & 2x  & 2x\\
2x & x-4 & 2x \\
 2x & 2x & x-4 
\end{vmatrix} =\brak{A+Bx}\brak{x-A}^2$, then the ordered pair \brak{A,B} is equal to: 
\hfill{[JEE M 2018]}
\begin{enumerate}
\begin{multicols}{4}
    \item \brak{-4,3}
    
    \item \brak{-4,5}
    \item \brak{4,5}
    \item \brak{-4,-5}
    \end{multicols}
\end{enumerate}
\item If the system of linear equations \\
\begin{align*}x+ky+3z=0 \\
3x+ky-2z=0 \\
2x+4y-3z=0 \end{align*}\\
has a non-zero solution \brak{x,y,z},then $\frac{xz}{y^2}$ is equal to :
\hfill{[JEE M 2018]}
\begin{enumerate}
\begin{multicols}{4}
    \item 10
    \item -30
    \item 30
    \item -10
    \end{multicols}
\end{enumerate}
\item The system of linear equations \\
\begin{align*}x+y+z=2 \\
2x+3y+2z=5 \\
2x+3y+(a^2-1)z=a+1 \end{align*} 
\hfill{[JEE M2019-9 Jan(M)]}
\begin{enumerate}[label=\alph*)]
    \item is consistent when $a=4$
    \item has a unique solution for $|a|= \sqrt{3}$
    \item has infinitely many solutions for $a=4$
    \item is consistent when $|a|= \sqrt{3}$
\end{enumerate}
\item If $A= \begin{bmatrix}
    \cos{\theta} & -\sin{\theta} \\
    \sin{\theta} & \cos{\theta}
\end{bmatrix}$, then the matrix $A^{-50}$ when $\theta=\frac{\pi}{12}$, is equal to: 
\hfill{[JEE M 2019-9 Jan(M)]}
\begin{enumerate}
\begin{multicols}{2}
    \item $\begin{bmatrix}
   \frac{1}{2}  & -\frac{\sqrt{3}}{2}  \\
    \frac{\sqrt{3}}{2} & \frac{1}{2}
\end{bmatrix}$ \\ 
\columnbreak
    \item $\begin{bmatrix}
   \frac{\sqrt{3}}{2}  & -\frac{1}{2}  \\
    \frac{1}{2} & \frac{\sqrt{3}}{2}
\end{bmatrix}$ \\ 
\end{multicols}
\begin{multicols}{2}

    \item $\begin{bmatrix}
   \frac{\sqrt{3}}{2}  & \frac{1}{2}  \\
    -\frac{1}{2} & \frac{\sqrt{3}}{2}
\end{bmatrix}$ \\

    \item $\begin{bmatrix}
   \frac{1}{2}  & \frac{\sqrt{3}}{2}  \\
    -\frac{\sqrt{3}}{2} & \frac{1}{2}
\end{bmatrix}$
\end{multicols}
\end{enumerate} 
\item If $\begin{bmatrix}
    1 & 1 \\
    0 & 1
\end{bmatrix}$.$\begin{bmatrix}
    1 & 2 \\
    0 & 1
\end{bmatrix}$.$\begin{bmatrix}
    1 & 3 \\
    0 & 1
\end{bmatrix}$\dots $\begin{bmatrix}
    1 & n-1 \\
    0 & 1
\end{bmatrix}$$=$$\begin{bmatrix}
    1 & 78 \\
    0 & 1
\end{bmatrix}$.\\
then the inverse of $\begin{bmatrix}
    1 & n \\
    0 & 1
\end{bmatrix}$ is 
\hfill{[JEE M2019-9 April(M)]} 
\begin{enumerate}
\begin{multicols}{2}
    \item $\begin{bmatrix}
        1 & 0 \\
        12 & 1 
    \end{bmatrix}$
    \columnbreak
    \item $\begin{bmatrix}
        1 & -13 \\
        0 & 1 
    \end{bmatrix}$
    \end{multicols}
    \begin{multicols}{2}
    \item $\begin{bmatrix}
        1 & -12 \\
        0 & 1 
    \end{bmatrix}$
     \item $\begin{bmatrix}
        1 & 0 \\
        13 & 1 
    \end{bmatrix}$   
    \end{multicols}
\end{enumerate}
\item Let $\alpha$ and $\beta$ be the roots of the equation $$x^2+x+1=0$$. Then for $y\ne0$ in R,\\
$\begin{vmatrix}
   y+1 & \alpha & \beta \\
    \alpha & y+\beta & 1 \\
    \beta & 1 & y+\alpha
\end{vmatrix}$ is equal to : 
\hfill{[JEE M 2019-9 April(M)]} 
\begin{enumerate}
\begin{multicols}{2}
    \item $y(y^2-1)$
    \columnbreak
    \item $y(y^2-3)$
    \end{multicols}
    \begin{multicols}{2}
    \item $y^3$
    \item $y^3-1$
    \end{multicols}
\end{enumerate}

\end{enumerate}
    
    
    
    
    \section*{Chapter 12 \\ Differentiation}



\section*{E : SUBJECTIVE PROBLEMS}
\begin{enumerate}
\item Let $f$ be a twice differentiable function such that 
$f''(x)=-f(x)$, and \begin{align*}f'(x)=g(x) , h(x)=[f(x)]^2+[g(x)]^2\end{align*}. Find $h(10)$ if $h(5)=11$ .
\hfill{(1982-3 Marks)}
\item If $\alpha$ be a repeated root of a quadratic equation $f(x)=0$ and $A(x),B(x)$ and $C(x)$ be polynomials of degree 3,4 and 5 respectively, then show that $\begin{vmatrix}
A(x) & B(x) & C(x) \\
A(\alpha) & B(\alpha) & C(\alpha) \\
A'(\alpha) & B'(\alpha) & C'(\alpha) 
\end{vmatrix}$ 
is divisible by $f(x)$, where prime denotes the derivatives.
\hfill{(1984-4 Marks)}
\item If $x=\sec{\theta}-\cos{\theta}$ and $y=\sec^n{\theta}-\cos^n{\theta}$, then show that $ \brak{x^2 + 4} \brak{\frac{dy}{dx}}^2=n^2 \brak{y^2+4}$ 
\hfill{(1989-2 Marks)}
\item Find $\frac{dy}{dx}$ at $x=-1$, when $(\sin{y})^{\sin(\frac{\pi}{2}x)} + \frac{\sqrt{3}}{2}\sec^{-1}{(2x)} + 2^x\tan{(\ln{(x+2)})} = 0 $
\hfill{(1991- 4 Marks)}
\item If $y = \frac{ax^2}{(x-a)(x-b)(x-c)}+\frac{bx}{(x-b)(x-c)}+1$, prove that $\frac{y'}{y}=\frac{1}{x}(\frac{a}{a-x}+\frac{b}{b-x}+\frac{c}{c-x})$
\hfill{(1998- 8 Marks)}
\end{enumerate}
\section*{H : Assertion \& Reason Type Questions}
\begin{enumerate}
    \item Let $f(x)=2 + \cos{x}$ for all real x.\\
    \textbf{STATEMENT - 1}:For each real $t$, there exists a point c in [t,t+$\pi$] such that $f'(c)=0$ because \\
    \textbf{STATEMENT - 2}: $f(t)=f(t+2\pi)$ for each real t.
    \hfill{(2007-3 Marks)}
    \begin{enumerate}[label=(\alph*)]
        \item Statement-1 is True, Statement-2 is True; Statement-2 is a correct explanation for Statement-1
        \item Statement-1 is True, Statement-2 is True; Statement-2 is NOT a correct explanation for Statement-1
        \item Statement-1 is True, Statement-2 is False 
        \item Statement-1 is False, Statement-2 is True
    \end{enumerate}
\item Let $f$ and $g$ be real valued functions defined on interval (-1,1) such that $g''(x)$ is continuous, $g(0)\neq0$.$g'(0)=0, g''(0)\neq0$ , and $f(x)=g(x)\sin{x}$\\
\textbf{STATEMENT-1}:\\
$\lim _{x \to 0}[g(x)\cot{x}-g(0)\csc{x}]=f''(0)$ and\\
\textbf{STATEMENT-2}: $f'(0)=g(0)$
\hfill{(2008)}
\begin{enumerate}[label=(\alph*)]
        \item Statement-1 is True, Statement-2 is True; Statement-2 is a correct explanation for Statement-1
        \item Statement-1 is True, Statement-2 is True; Statement-2 is \textbf{NOT} a correct explanation for Statement-1
        \item Statement-1 is True, Statement-2 is False 
        \item Statement-1 is False, Statement-2 is True
    \end{enumerate}
\end{enumerate}
\section*{I:Integer Value Correct Type}
\begin{enumerate}
    \item If the function $f(x)=x^3+e^{\frac{x}{2}}$ and $g(x)=f^{-1}(x)$, then the value of $g'(1)$ is 
    \hfill{(2009)}\\
    \item    Let $f(\theta)=\sin{\brak{\tan^{-1}{\brak{\frac{\sin{\theta}}{\sqrt{\cos{2\theta}}}}}}}$ , where $-\frac{\pi}{4}<\theta<\frac{\pi}{4}$. Then the value of $\frac{d}{d(\tan{\theta})}(f(\theta))$ is
\end{enumerate}  
\hfill{(2011)}


\end{document}
