%iffalse
\let\negmedspace\undefined
\let\negthickspace\undefined
\documentclass[journal,,12pt,onecolumn]{IEEEtran}
\usepackage{cite}
\usepackage{amsmath,amssymb,amsfonts,amsthm}
\usepackage{algorithmic}
\usepackage{graphicx}
\usepackage{textcomp}
\usepackage{xcolor}
\usepackage{txfonts}
\usepackage{listings}
\usepackage{enumitem}
\usepackage{mathtools}
\usepackage{gensymb}
\usepackage{comment}
\usepackage[breaklinks=true]{hyperref}
\usepackage{tkz-euclide} 
\usepackage{listings}
\usepackage{gvv}                                        
%\def\inputGnumericTable{}                                 
\usepackage[latin1]{inputenc}                                
\usepackage{color}                                            
\usepackage{array}                                            
\usepackage{longtable}                                       
\usepackage{calc}                                             
\usepackage{multirow}                                         
\usepackage{hhline}                                           
\usepackage{ifthen}                                           
\usepackage{lscape}
\usepackage{tabularx}
\usepackage{array}
\usepackage{float}
\usepackage{multicol}



\newtheorem{theorem}{Theorem}[section]
\newtheorem{problem}{Problem}
\newtheorem{proposition}{Proposition}[section]
\newtheorem{lemma}{Lemma}[section]
\newtheorem{corollary}[theorem]{Corollary}
\newtheorem{example}{Example}[section]
\newtheorem{definition}[problem]{Definition}
\newcommand{\BEQA}{\begin{eqnarray}}
\newcommand{\EEQA}{\end{eqnarray}}
\newcommand{\define}{\stackrel{\triangle}{=}}
\theoremstyle{remark}
\newtheorem{rem}{Remark}

% Marks the beginning of the document
\begin{document}
\bibliographystyle{IEEEtran}
\vspace{3cm}

\title{XE\\2007}
\author{EE24BTECH11063 - Y.Harsha Vardhan Reddy}
\maketitle

\bigskip

\renewcommand{\thefigure}{\theenumi}
\renewcommand{\thetable}{\theenumi}

\section*{B : Computational science}
\subsection*{Q.7 - Q.24 carry two marks each}
\begin{enumerate}
    \item The minimum number of terms required in the series expansion of $e^x$ to evaluate at $x=1$ correct up to 3 places of decimals is
    \begin{enumerate}
        \begin{multicols}{4}
            \item 8
            \item 7
            \item 6
            \item 5
        \end{multicols}
    \end{enumerate}
    \bigskip
\item The iteration scheme $x_{n+1}=1/\brak{1+x_n}^2$ converges to a real number $x$ in the interval $\brak{0,1}$ with $x_0=0.5$. The value of $x$ correct up to 2 places of decimal is equal to
\begin{enumerate}
    \begin{multicols}{4}
        \item 0.65
        \item 0.68
        \item 0.73
        \item 0.80
    \end{multicols}
\end{enumerate}
\bigskip
\item If the diagonal elements of a lower triangular square matrix A are all different from zero, then the matrix A will always be
\begin{enumerate}
    \begin{multicols}{4}
        \item symmetric
        \item non-symmetric
        \item singular
        \item non-singular
    \end{multicols}
\end{enumerate}
\bigskip
\item If two eigen values of the matrix
\begin{align*}
    M=\myvec{2&6&0\\1&p&0\\0&0&3}
\end{align*}
are -1 and 4, then the value of $p$ is
\begin{enumerate}
    \begin{multicols}{4}
        \item 4
        \item 2
        \item 1
        \item -1
    \end{multicols}
\end{enumerate}
\bigskip
\item Consider the system of linear simultaneous equations
\begin{align*}
    x+10y=5;\;y+5z=1;\;10x-y+z=0
\end{align*}
On applying Gauss-Seidel method the value of x correct up to 4 decimal places is
\begin{enumerate}
    \begin{multicols}{4}
        \item 0.0385
        \item 0.0395
        \item 0.0405
        \item 0.0410
    \end{multicols}
\end{enumerate}
\bigskip
\item The graph of a function $y=f\brak{x}$ passes through the points $\brak{0,-3},\;\brak{1,-1}\text{and} \brak{2,3}$. Using Lagrange interpolation, the value of $x$ at which the curve crosses the x-axis is obtained as
\begin{enumerate}
    \begin{multicols}{4}
        \item 1.375
        \item 1.475
        \item 1.575
        \item 1.675
    \end{multicols}
\end{enumerate}
\bigskip
\item The equation of the straight line of best fit using the following data
\[
\begin{array}{|c|c|c|c|c|c|}
\hline
x & 1 & 2 & 3 & 4 & 5 \\
\hline
y & 14 & 13 & 9 & 5 & 2 \\
\hline
\end{array}
\]
by the principle of least square is
\begin{enumerate}
    \begin{multicols}{2}
        \item $y=18-3x$
        \columnbreak
        \item $18.1-3.1x$
        \end{multicols}
        \begin{multicols}{2}
        \item $y=18.2-3.2x$
        \item $18.3-3.3x$
    \end{multicols}
\end{enumerate}
\bigskip
\item On solving the initial value problem\\
$\frac{dy}{dx}=xy^2,\; y\brak{1}=1$ by Euler's method, the value of $y$ at $x=1.2$ with $h=0.1$ is
\begin{enumerate}
    \begin{multicols}{4}
        \item 1.1000
        \item 1.1232
        \item 1.2210
        \item 1.2331
    \end{multicols}
\end{enumerate}
\bigskip
\item The local error of the following scheme\\
$y_{n+1}=y_n+\frac{h}{12}\brak{5y'_{n+1}+8y'_{n}-y'_{n-1}}$\\
by comparing with the Taylor series $y_{n+1}=y_n+hy'_n+\frac{h^2}{2!}{y^n}_n+\cdots$ is
\begin{enumerate}
    \begin{multicols}{4}
        \item $O\brak{h^4}$
        \item $O\brak{h^5}$
        \item $O\brak{h^2}$
        \item $O\brak{h^3}$
    \end{multicols}
\end{enumerate}
\bigskip
\item The area bounded by the curve $y=1-x^2$ and the x-axis from $x=-1$ to $x=1$ using Trapezoidal rule with step length $h=0.5$ is
\begin{enumerate}
    \begin{multicols}{4}
        \item 1.20
        \item 1.23
        \item 1.25
        \item 1.33
    \end{multicols}
\end{enumerate}
\bigskip
\item The iteration scheme\\
$x_{n+1}=\sqrt{a}\brak{1+\frac{3a^2}{{x_n}^2}}-\frac{3a^2}{x_n},\; a>0$ converges to the real number
\begin{enumerate}
    \begin{multicols}{4}
        \item $\sqrt{a}$
        \item $a$
        \item $a\sqrt{a}$
        \item $a^2$
    \end{multicols}
\end{enumerate}
\bigskip
\item If the binary representation of two numbers $m$ and $n$ are 01001101 and 00101011, respectively, then the binary representation of $m-n$ is 
\begin{enumerate}
    \begin{multicols}{4}
        \item 00010010
        \item 00100010
        \item 00111101
        \item 00100001
    \end{multicols}
\end{enumerate}
\bigskip
\item Which of the following statements are true in a C program?\\
P: A local variable is used only within the block where it is defined, and its sub-blocks\\
Q: Global variables are declared outside the scope of all blocks\\
R: Extern variables are used by linkers for sharing between other compilation units\\
S: By default, all global variables are extern variables
\begin{enumerate}
  \begin{multicols}{4}
        \item P and Q
        \item P,Q and R
        \item P,Q and S
        \item P,Q,R and S
    \end{multicols}
\end{enumerate}
\bigskip
\item The iteration scheme\\
$x_{n+1}=\sqrt{a}\brak{1+\frac{3a^2}{{x_n}^2}}-\frac{3a^2}{x_n},\; a>0$ converges to the real number
\begin{enumerate}
    \begin{multicols}{4}
        \item $\sqrt{a}$
        \item $a$
        \item $a\sqrt{a}$
        \item $a^2$
    \end{multicols}
\end{enumerate}
\bigskip
\item Consider the following recursive function $g\brak{}$
\lstset{language=[90]Fortran,
        basicstyle=\ttfamily,  % Set basic style to typewriter font, all in black
        keywordstyle=\ttfamily,  % No color for keywords
        commentstyle=\ttfamily,  % No color for comments
        %numbers=left,  % Remove line numbering
        numberstyle=\tiny,
        stepnumber=1,
        numbersep=5pt,
        showstringspaces=false,
        tabsize=4,
        breaklines=true,
        frame=none}  % Removes the box

\begin{lstlisting}
Recursive integer function g(m,n) result(r)
    integer:: m, n
    if (n == 0) then
        r = m
    else if (m <= 0) then
        r = n+1
    else if ((n - n/2*2) == 1) then
        r = g(m-1, n+1)
    else
        r = g(m-2, n/2)
    end if
end
\end{lstlisting}
Which value will be returned if the function g is called with 6,6?
\begin{enumerate}
    \begin{multicols}{4}
        \item 2
        \item 4
        \item 6
        \item 8
    \end{multicols}
\end{enumerate}
\bigskip
\item If the following function is called with $x=1$
\lstset{language=[90]Fortran,
        basicstyle=\ttfamily,  % Set basic style to typewriter font, all in black
        keywordstyle=\ttfamily,  % No color for keywords
        commentstyle=\ttfamily,  % No color for comments
        %numbers=left,  % Remove line numbering
        numberstyle=\tiny,
        stepnumber=1,
        numbersep=5pt,
        showstringspaces=false,
        tabsize=4,
        breaklines=true,
        frame=none}  % Removes the box
\begin{lstlisting}
real function print_value(x)
real:: x, sum, term
integer:: i
i = 0
sum = 2.0
term = 1.0
do while (term > 0.0001)
term = x * term/(i+1)
sum = sum + term
i = i + 1
end do
print_value = sum
end
\end{lstlisting}
the value returned will be close to
\begin{enumerate}
    \begin{multicols}{4}
        \item $\log_e{2}$
        \item $\log_e{3}$
        \item $1+e$
        \item $e$
    \end{multicols}
\end{enumerate}
\bigskip
\item Consider the following C program
\lstset{language=[90]Fortran,
        basicstyle=\ttfamily,  % Set basic style to typewriter font, all in black
        keywordstyle=\ttfamily,  % No color for keywords
        commentstyle=\ttfamily,  % No color for comments
        %numbers=left,  % Remove line numbering
        numberstyle=\tiny,
        stepnumber=1,
        numbersep=5pt,
        showstringspaces=false,
        tabsize=4,
        breaklines=true,
        frame=none}  % Removes the box
\begin{lstlisting}
#include <stdio.h>
#include <string.h>
void main()
{
char s[80], *p;
int sum = 0;
p = s;
gets(s);
while (*p)
{
if (*p == '1')
sum = 2*sum +1;
else if (*p == '0')
sum = sum*2;
else
printf("invalid string");
p++;
}
printf("%d", sum);
}
\end{lstlisting}
Which number will be printed if the input string is 10110?
\begin{enumerate}
    \begin{multicols}{4}
        \item 31
        \item 28
        \item 25
        \item 22
    \end{multicols}
\end{enumerate}
\bigskip

\end{enumerate}
\end{document}

